\begin{abstract}
For verification to make the leap from the laboratory to
practical use,
it must become performant, maintainable, and accessible.


\end{abstract}

\section)Introduction}
Systems researchers have begun to demonstrate that substantial
systems software components can be constructed using verification
as the primary means of establishing functional correctness.
Compared to ordinary test-driven software development,
verfication promises effectily exhaustive testing.

What stands in the way of verification replacing testing?
Verification must produce software that offers competitive \textit{performance}.
The languages and techniques must be \textit{accessible}
to mainstream developers.
The burden of creating verified software and its associated
proof-guiding annotations must be \textit{affordable}.
The proof artifact must be incrementally \textit{maintainable} to evolve
as the specification and implementation code evolve.

We present {\veribetrkv}, a high-performance key-value store
with the explicit goal of developing a maintainable code artifact
with a verification methodology.
{\veribetrkv} implements the persistent key-value storage substrate of
BetrFS\cite{}.
We build {\veribetrkv} as the first step in building a verified BetrFS,
with the explicit goal of making an adoptoble filesystem:
one \texit{faster} than ext4, free of corruption and crash-fault bugs,
and maintainable.
On realistic workloads it outperforms RocksDB over disk-like devices
and is competitive within 30\% even over emerging nonvolatile storage.

The methodology, called {\ironstar},
builds on and improves the accessibility and automation of
the tools and techniques used in prior projects.

The executable code in {\ironstar} is specified as imperative, heap-mutating
code, giving the developer a high degree of control over memory behavior
and hence system perforamnce.
{\ironstar} prescribes a refinement-based plan for organizing proof code
into localized modules that enhance maintainability.

To be affordable, a verification methodology must heavily exploit automation.
{\ironstar} improves on its methodological predecessor by offering a path
exploiting maximal automation while avoiding the timeouts that have plagued
prior highly-automated approaches.

The evaluation demonstrates that {\veribetrkv} performs comparably to
state-of-the-art unverified key-value storage, and dramatically outperforms
prior verified systems when considering realistic workloads that stress
the filesystems' CPU activity.

We show that changes to {\veribetrkv} implementation create isolated
changes to its proof obligations, showing that {\ironstar} code is
refactorable and maintainable.

{\veribetrkv} demonstrates these attributes while maintaining an affordable
burden as compared with prior approaches, as measured by the proof:code
ratio.

\endsection{Introduction}
